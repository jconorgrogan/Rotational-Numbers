\documentclass{article}
\usepackage{amsmath, amssymb}

\begin{document}

\section*{Mathematical Descriptions}

\subsection*{Foundation}
Mathematical entities emerge from infinity, denoted as \( \infty \). Consider \( \infty \) as a vector space over which all structures are spanned. Dividing or partitioning this infinite expanse, perhaps symbolically as segments \([0, \infty/n]\), gives rise to discernible mathematical substructures. Using geometric and topological spaces, such as the \( p \)-adic metric space, these partitions can elucidate number properties, like prime densities.

\subsection*{Divisions of Infinity}
Partitioning infinity can be visually mapped to the unit circle \( S^1 \) in the complex plane, where each partition corresponds to an \( e^{i\theta} \). This provides a tangible representation to the infinite concept.

\subsection*{Role of Primes}
Let \( p \) be a prime. Infinity divided by primes creates spaces that can be embedded in infinite-dimensional Hilbert spaces. For instance, \( \infty \) partitioned by 2 or 3 might introduce structures that resemble binary trees or ternary Cantor sets, respectively.

\subsection*{Composites and Their Geometry}
Let a composite \( c \) have prime factorization \( p_1^{n_1}p_2^{n_2}\ldots \). Geometrically, a composite can be seen as a product space of its prime components' structures. The number 4, or \( 2^2 \), induces a "dual duality", visualized as a bifurcation of a binary tree.

\subsection*{Geometric Symmetry}
Representing these partitions as radial lines or chords on \( S^1 \), primes like \( p \) introduce symmetry groups, \( G_p \), that possess unique maximal symmetries. These can be considered under group actions that preserve certain structures, like a dihedral group action. Composites, however, give rise to symmetries that are direct products of their prime factors' symmetry groups.

\subsection*{Angle Doubling and Wave Dynamics}
The transformation \( f(\theta) = 2\theta \mod 2\pi \) when acted on functions, say \( g(e^{i\theta}) \), on the unit circle produces modulated versions. These can be seen as wavelets, forming intricate patterns due to interference, akin to wave superposition. The frequency doubling transformation, in the context of Fourier analysis, acts as a dilation operator on Fourier modes, amplifying certain spectral components.

\end{document}
